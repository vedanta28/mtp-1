Cyber-Physical Systems (CPS) are integral to a wide range of critical applications, including autonomous vehicles, medical devices, and industrial automation. These systems merge physical processes with computational elements, creating complex environments that require stringent safety, security, and performance guarantees. However, the dynamic nature of CPS, combined with unpredictable environmental interactions, introduces significant challenges in ensuring system correctness at runtime. \\

\noindent Runtime enforcement plays a crucial role in addressing these challenges by monitoring and dynamically adjusting the system's behavior to adhere to predefined safety and security policies. Unlike static verification techniques, which may not account for unforeseen scenarios or evolving threats, runtime enforcement provides a real-time safeguard that detects and mitigates violations as they occur. This proactive approach enhances the resilience of CPS, ensuring they remain robust against internal faults and external attacks, ultimately protecting both the system's operation and its users. \\

\noindent In this work, we explore the use of compositional runtime enforcement in underwater robotic swarms, comparing its effectiveness against the traditional monolithic approach. The study focuses on applying runtime enforcement to a Multi-Robot Coverage Path Planning (MRCPP) algorithm, specifically designed for mapping underwater vegetation within a seagrass bed. By examining both approaches, we aim to assess their impact on the algorithm's performance, scalability, and adaptability in dynamic underwater environments.

\newpage
%\null\newpage